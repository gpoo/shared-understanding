\chapter{Conclusion and Future Work}

In this thesis I proposed a theory of coordination and communication in software organizations based on the principle of shared understanding. I argued that coordination and communication are both essential and problematic issues in modern software development, and that a naive and simplistic conceptualization of these terms has led us to an unproductive state of research. I claimed that it is possible and useful to bring a more complex conceptualization of coordination and communication, and I used the concept of shared understanding to tie both elements together. In this way, I defined coordination as the development and negotiation of a shared understanding of goals and plans among members of an organization, and communication as the development of a shared understanding of status and context. I explained that such a shared understanding is more easily established when participants interact with synchrony, proximity, proportionality, and maturity.

I discussed how this theoretical stance brings several relevant implications for the study and practice of software development. The theory of shared understanding implies that project lifecycle processes and documentation are poor substitutes for simpler, cheaper, and more informal but less scalable coordination and communication mechanisms, such as situation-appropriate sets of practices and tools. It also implies that physical space is an important enabler of shared understanding attributes, and that organizational growth leads to the inefficient formalization of organizational dynamics and renders some shared understanding strategies unapplicable.

The theory is supported by interdisciplinary theoretical and empirical developments and by my own empirical data obtained from five case studies. However, the theory as currently stated still leaves many open questions and opportunities for future research. I believe the following projects would be worth conducting to probe these questions further.

First, there is an unresolved relationship to the socio-technical congruence body of work. In the near future, I will develop and evaluate new procedures to measure Socio-Technical Congruence (STC) that take into account rich and informal interactions that are ignored with current metrics. The STC community has already explored a number of alternatives to measure the fitness between software products and the organizations that produced them. These alternatives depend on mining software repositories, and although they have yielded important findings, they are limited due to the fact that repositories do not capture the most significant determinants of social structure in organizations: rich, informal, usually face-to-face interactions. This research would improve on current mechanisms to measure fitness by including heuristics that take into account geographical, social, and cultural factors of organizations, in order to produce fitness metrics that are more reliable and have greater explanatory power than the current ones, and that can be used both by researchers and by software development teams. The new measures will be compared to existing ones in terms of their relationship to current performance measures, such as integration failures and number of defects.

Second, I will perform a longitudinal study of roles and responsibilities in small software groups. Roles and responsibilities are, currently, blind spots in software development research: we speak of ``developers,'' ``testers,'' or some other roles without considering what they entail, nor their interactions and linkages beyond the most superficial level. The study I propose involves collecting field data from a number of small software groups (both for-profit and otherwise), in several interventions spanning a period of at least one and a half years, to discover the roles they carry out in practice and the interactions between them. The analysis of this data could follow the techniques developed by Breiger and Pattison \shortcite{Breiger1986}. This research will be based on the principle that responsibilities cluster in roles based on the cohesion between said responsibilities; a good characterization of role interaction would lead to the creation of better role definitions and growth paths for software development organizations. The outcome of this project will be a model of roles involved in real-world software development and of the interaction between these roles in routine and extraordinary situations (using the term ``situations'' as intended in our theory).

Third, although the domain of this thesis is software organizations, its main arguments are in principle applicable to any group of people performing intellectually demanding work with an exploratory or creative design component, tacit and complex knowledge, and an abundance of information. However, none of the studies I have conducted so far reaches out of the software development activity. I plan to perform a comparison to coordination and communication in teams of other domains with these characteristics to evaluate whether the arguments of the theory hold across them.

Finally, I will develop and evaluate an instrument to measure indices of shared understanding robustness in software development teams. This thesis discussed these attributes at relatively abstract levels: we can point to instances of their presence, but we cannot perform reliable comparisons between organizations or within an organization at several points in time. The instrument I propose would provide a convenient mechanism to perform these comparisons, and it would provide guidelines for software development teams wishing to improve their performance through a more robust development of shared understanding. I will collect feedback from practitioners that will use this instrument to see if and how it helped them draw recommendations on improvement based on their results and our guidelines.

I do not think that this work is finished; this thesis is only one step towards research that gives coordination and communication problems the attention they deserve. The projects described above are a sample of the kind of work that this theory leads to, but there are many other possibilities, especially with respect to the validation or refutation of the theory, that are appealing and, I hope, fruitful for the development of our field.
