\chapter{Introduction}

This thesis addresses the problem of achieving effective coordination and communication dynamics in software organizations. Recognition of the importance of this problem has been growing in recent years, although, as we will see, current strategies to attack it are disparate and conceptually confused. This confusion is at least partly caused by ambiguities in our understanding of what it means to coordinate and to communicate, as well as a disagreement on the appropriate social structures on which to focus our studies.

We speak of coordination often, and most of us can distinguish between extremely coordinated and uncoordinated groups, but our definitions for coordination are rather unsatisfactory. Accordingo to Malone and Crowston \shortcite{Malone1994}, for instance, coordination consists of \emph{``managing dependencies between activities.''} Presumably, in the context of software development, the ``activities'' are all the tasks that stakeholders need to perform in order to produce a satisfactory software artifact. These activities have ``dependencies'' in the sense that the characteristics or the completion of each of them influence the others, and these dependencies need to be ``managed.'' This definition raises questions regarding what happens when the activities and their dependencies are not known, what does management consist of, and what is the goal of this management exercise. Other definitions emphasize different constructs and theoretical stances, leading to significant discrepancies in our understanding of the concept.\footnote{Malone and Crowston's survey offers a collection of these definitions. For example: Coordination as ``the emergent behaviour of collections of individuals whose actions are based on complex decision processes,'' as ``information processing within a system of communicating entities with distinct information states,'' and as ``composing purposeful actions into larger purposeful wholes.''} As a result of these ambiguities, most researchers in our area seem to make some implicit assumptions about what it means to coordinate, and since their assumptions are not always compatible, they reach very different positions and recommendations.

Communication, in turn, is often treated as a fairly mechanistic construct in our community: a problem of \emph{``reproducing at one point either exactly or approximately a message selected at another point''} \cite{Shannon1948}. Though more sophisticated and appropriate conceptualizations of communication are available \cite{Littlejohn2002}, Reddy \shortcite{Reddy1993} shows that this basic model is linguistically prevalent in our society, and Bryant \shortcite{Bryant2000} argues that it is firmly established in software research. This model relies on a questionable but common metaphor of communication that portrays information as a thing, or rather a fluid; a substance that flows through abstract channels and that can be passed on between individuals, or from individuals to artifacts and back to other individuals. I will argue that this metaphor is inadequate for the kinds of communication that software development requires, and so, it would seem, much of our research is grounded on a flawed understanding of one of its basic elements.

Both concepts, coordination and communication, are closely related in our context. Communication enables coordination, since it is impossible to coordinate without at least a modicum of communication.\footnote{The argument holds even in situations where no communication is necessary during the task: such situations depend on communication done in advance, so the relevant parties know that silence, in a particular context, carries some meaning.} In turn, coordination builds and refines the conventions through which communication takes place. In this thesis I will argue that coordination and communication are conceptually separate but practically intertwined activities, and I will offer what I think are better and more useful conceptualizations of both activities than those available until now in our literature.

There is a tendency in the software research community to study these topics at the level of idealized software development teams. Such teams, typically composed of a number of developers, testers, and project managers, are treated as if they were isolated from the social structures in which they are embedded, as if the people that do not directly contribute to the generation or testing of code were out of bounds for our research. In contrast, this thesis addresses the problem at the level of the software organization, which does include one or several teams in charge of developing a software product but also the staff that supports their activities and the people with a stake in the survival and viability of the organization. This allows me to include in my analysis some economic, bureaucratic, and strategic considerations that often drive technical decisions but that tend to be left aside in software development research.

The problem of achieving effective coordination and communication is pervasive in software development. It manifests itself in several forms in our field. For instance, many requirements errors are caused by faults in coordination or communication, and they are known to be numerous \cite{Jones1996}, dangerous \cite{Neumann1995}, and extremely expensive errors \cite{Boehm1988}. Similarly, integration errors are often manifestations of earlier coordination and communication mistakes, and are also known to doom or impair software projects that seemed healthy until their last moments.

Arguably, although some projects attempt the creation of products of extremely sophisticated technical or computational complexity, the great majority of software projects attempted today are technically feasible or even straightforward; they work with moderately reliable hardware and within the confines of settled Computer Science knowledge. For the people working on such projects the main challenge consists of performing highly specialized, sensitive, creative, and professional work \emph{as part of a team}: the team has to discover all the relevant aspects of the problem it intends to solve, and it has to find efficient mechanisms to coordinate the team members' efforts to solve it. No amount of technical sophistication can overcome the problem of an organization whose members cannot coordinate or communicate between themselves.


\section{Historical development}

The problem of achieving effective coordination and communication dynamics has long been recognized as one of the central problems of our field. The earliest reference to this problem seems to be Conway's \shortcite{Conway1968} observation of the practical impact of communication dynamics in software teams. He observed that ``any organization which designs a system will inevitably produce a design whose structure is a copy of the organization's communication structure.'' His observation (which eventually came to be known as Conway's Law) was meant to address any kind of design activity, not just software development, but the intensive and large-scale challenges of system design in software projects made it particularly relevant to our field. Conway's Law was important both for practitioners and researchers: it entailed that crafting good software requires team structures that fit the task, as well as communication dynamics that effectively allow people in these social structures to access the information that they need.

Years later, in \emph{The Mythical Man-Month}, Brooks \shortcite{Brooks1975} noted that the prevalent practice of measuring the effort to develop software in man-month units is fallacious because software development depends heavily on coordination and communication demands, and these demands could potentially grow polynomially with the size of the team. It is easy to see why. A group of two people needs to maintain only one communication channel: the one which allows them to coordinate with each other. But a group of five has ten channels to maintain (since the number of channels equals the maximum number of edges in their social network), and a group of fifty has 1,225. Since each of these channels requires additional effort in the form of coordination activities, then one cannot expect to add people to a project and decrease the project's completion date at a constant rate. In practice, of course, team members do not need to coordinate directly with every other peer. Through proper organizational structuring and software modularization, the number of active communication channels is usually far lower than the theoretical maximum. But this does not detract from Brooks' warning: coordination demands in software teams are so intense that, unchecked, they may cause an addition of labour to result in an overall loss of productivity.

These coordination and communication concerns came to the forefront with the publication of \emph{Peopleware} \cite{DeMarco1987}, a book on project management written by two software consultants who claimed that ``the major problems of [software development] work are not so much \emph{technological} as \emph{sociological} in nature.'' Presented as a series of essays backed by a combination of anecdotal experiences and empirical studies, \emph{Peopleware} argued for the relevance of concepts such as team ``jelling,'' morale, and the proactive use of office space. The book spurred a strong and sympathetic reaction by a community that felt misunderstood and mistreated by corporations with a technology- or process-centric view of software development, and it continues to be popular today.

Despite these early works, coordination and communication remained minority concerns within the software research community, as evidenced by the small proportion of papers that explore the subject in the leading conferences and journals of the field. Beyond some ubiquitous lip service to the idea that ``people issues are important,'' few researchers focused on them before the year 2000. Among those relevant to our topic, Curtis \emph{et al.}\ \shortcite{Curtis1988} reported that breakdowns in communication and coordination are a major problem in software development; Walz \emph{et al.}\ \shortcite{Walz1993} described the progression of shared understanding through a series of design meetings in a software team; Perry \emph{et al.}\ \shortcite{Perry1994} found that software developers spend over half of their working time communicating; Kraut and Streeter \shortcite{Kraut1995} argued that although formal communication procedures had appropriate tool support, informal communication procedures needed to be nurtured as well; and Herbsleb and Grinter \shortcite{Herbsleb1999} reported that geographical distance adds great challenges to coordination because of the corresponding breakdown of informal communication channels.

Since the turn of the century, there has been a growing interest in exploring coordination and communication problems in software research. This is evidenced by the increasing amount of relevant work presented at gatherings such as the Coordination and Human Aspects in Software Engineering workshop and the Socio-Technical Congruence workshop, and by the now frequent incursions into these topics at the main Software Engineering conferences and journals. The next chapter discusses some of this recent literature.

These studies are valuable on their own, but collectively they present a problem. They are conceptually and practically disconnected, each report an island. Perhaps this is so because coordination and communication are not common concerns among computer scientists, or because there is an overwhelming variety of academic disciplines that tackle these issues in different ways, and achieving mastery over any of them appears to hinder one's efforts for achieving mastery in one's own domain (and even when a number of scientists have individually succeeded in their interdisciplinary efforts, there is no guarantee that all of them will be drawing from compatible sociological, psychological, or organizational theories and constructs). But regardless of its causes, the consequences of this isolation are clear. Studies on coordination and communication in software organizations tend to lack concrete theoretical foundations. Their constructs are idiosyncratic and they can rarely be compared with those of other studies with any usefulness. Their findings are also isolated; there is no sense of a progressive improvement over previous models formulated and accepted by the community.

The problem is amplified by our community's disdain of theoretical work \cite{Hannay2007}: software researchers do not usually articulate theories to explain software development practice \cite{Jorgensen2004}. We have collected some insights, such as the observations by Conway and Brooks described above, but there are still no principles that allow us to link them with an explicit, unifying theoretical framework.


\section{Overview of the argument}

In this thesis I propose a theory that uses the concept of ``shared understanding'' as the key to explain the problems and solutions of coordination and communication in software organizations. Broadly, the argument of the thesis consists of the following points. First, it shows that achieving effective coordination and communication dynamics, though essential for the success of most software projects, is a particularly difficult problem due to (1) an overwhelming amount of information that multiple team members need to parse, (2) the need for the development of tacit, complex, and specialized knowledge, and (3) the exploratory nature of software development. I then demonstrate that for many of the events that matter in a software project, effective coordination and communication cannot be reliably achieved through the simple transfer of data among people and between people and artifacts. A deeper understanding of the situations in which an organization acts, and of the strategies that the organization will deploy, must be shared by its members. I argue that the members of software organizations face a constant struggle to share and negotiate an understanding about their goals, plans, status, and context, and that this struggle lies at the core of their coordination and communication problems. I then claim that those organizations that have values, structures, and practices which facilitate the development of this shared understanding will find it easier to coordinate and communicate effectively, achieving greater success in their software projects.

This thesis reframes the concepts of coordination and communication to better suit our needs, using the concept of shared understanding as their unifying link. It argues that an understanding of goals, plans, status, and context is more easily shared and negotiated when the mechanisms used for this purpose are \emph{synchronous}, \emph{proximate}, \emph{proportionate}, and \emph{mature}. The theory that results from these considerations has important implications for our field. It implies that project lifecycle processes and documentation are poor mechanisms to coordinate and to communicate because they are usually deficient across the above criteria. However, for all their deficiencies, processes and documentation may be the only feasible mechanisms in some circumstances, given that their substitutes, though more desirable, are often less scalable. Among these substitutes I discuss some software development practices and tools, the development of group cohesion, physical co-location, and reining in organizational growth. The theory implies that practices and tools are valuable to the extent that they enable the development of a shared understanding, irrespective of their other benefits, that group cohesion and physical co-location are powerful enablers of shared understanding dynamics, and that, since organizational growth leads to formalization and a loss of cohesion, it is harmful, and based purely on these considerations it should be avoided.

These implications are translatable into testable hypotheses, so the theory is refutable and has predictive power. Furthermore, given that they are relevant to many aspects of software research and practice, the theory is useful and directly applicable in our field. Even if other researchers reject the framework I propose here, it should bring forward a stronger foundational dialogue for the study of coordination and communication in software development.

The rest of the thesis proceeds as follows. Chapters 2 and 3 summarize the work that is relevant to the theory within our own community and from other disciplines, respectively. Chapter 4 discusses the methods used in my empirical work and presents the case studies that inform the theory. Chapter 5 states the theory, explores its implications, and offers a view of how software research and practice may be shaped by its arguments. Chapter 6 discusses the theory's practical and theoretical challenges, as well as its relationships with other current efforts to study coordination and communication in our field. Finally, Chapter 7 concludes the argument with a list of open questions and a description of the work I intend to do to pursue these questions further in the short and long term.
