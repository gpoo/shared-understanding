\chapter{Methods and Evidence}
\label{chap:Methods}

The arguments in the following chapter are supported with findings from five different studies. This chapter discusses the methodological choices I made for those studies and provides a brief description of each of them. Papers based on all but the most recent study have been published. I discuss the studies in depth in the appendices. The next chapter refers to all of them when presenting particular pieces of evidence.


\section{Case Study Method}

All of the studies on which this thesis is based follow the case study methodology \cite{Yin2003} to some extent. There are several reasons for this choice.

First, the case study methodology allowed me to focus on variables and constructs that cannot be studied appropriately in controlled laboratory experiments \cite{Easterbrook2007}. This thesis deals with self-interested and group-interested interactions among individuals which form part of the same organization and purportedly share some goals and benefits from the long-term success of their group.

The replication of such conditions in a controlled environment is notoriously difficult, if not impossible; certainly it has not been achieved to date in our domain \cite{Lung2008}. Several recent studies, notably from the Simula Research Laboratory, use hundreds of paid professionals as their participants, achieving a better reliability than the typical experiments in our field (which suffer from poor sample sizes or student subjects), but still failing to address the problems generated by extracting participants from their natural settings to perform artificially in a controlled environment. A telling example is a recent evaluation of pair \emph{versus} solo programming \cite{Arisholm2007}. The experiment involved the participation of nearly three hundred subjects for a day; it concluded that pair programming shows no improvements in time or quality, and it causes a net increase in the effort required to produce the same software as that of solo programmers. The study's design and execution are almost impeccable, but its findings regarding pair programming, unfortunately, cannot be applied beyond the very narrow context of pairs of strangers coming together to work on a new problem for a single day. Pair programming is not supposed to yield instant benefits; it is supposed to help teams of developers to familiarize with each other's strengths and weaknesses, to allow pairs to consider both the low-level details of implementation and the high-level architectural implications of their decisions simultaneously, and to provide a fail-safe mechanism in the event that the team loses a member: there is always another person who is familiar with the code the lost member created. Pair programming is also intended to be used in conjunction with other complementary practices that improve job satisfaction and lead to a number of benefits, including a decreased turnover and a greater vested interest in the success of the pair's project \cite{Beck1999}. It is possible that pairs performing in the way that pair programming is prescribed may excel over solo programmers, but the Arisholm \emph{et al.}\ study cannot offer any insight into this possibility.

A second reason to use case studies is that they aim to create, strengthen, and reject theories, in contrast with other qualitative methods that shy away from abstraction and generalization. In this sense they seem more useful for a problem with practical implications such as the one that concerns us, as they are designed to address the widespread lack of theoretical foundations that afflicts our discipline \cite{Hannay2007}.

Finally, a third reason to use case studies is their flexibility in data collection and analysis strategies. Mixed methods are welcome in case studies \cite{Yin2003}. Where applicable I included surveys as part of my study design, and the results from these surveys were used to strengthen or question some of the implications from my interviews and observations.

I have not explored this flexibility to its full potential yet. For instance, there are some settings where the addition of data mining methods to a combination of observations and interviews would be appropriate. I did not pursue this path---in fact I have spoken against an overreliance in data mining approaches in our field \cite{Aranda2009}. But data mining may yield significant benefits if used as a complement of other research methods. I discuss this possibility in greater detail in the final chapters of this thesis.


\section{Organizational Focus}

Among the many possible units of analysis in the study of coordination and communication for software development, the most commonly used seem to be individual developers \cite{Ko2007} and project teams \cite{Teasley2002}. However, while these units of analysis are valuable, they present a decontextualized picture of software development, in detriment to their findings.

It should be easy to see that, for the study of coordination and communication, a focus on teams is usually better than a focus on individuals. Software development is almost never conducted by single individuals in isolation from their environment; the teams in which said individuals perform affect their expectations, goals, habits, abilities, strategies, preferences, and culture. A team focus allows researchers to be more perceptive to these influences. It also allows them to consider the kind of software the team builds, the kind of clients it has, and the shared history of collaboration of its members. All of these factors may have significant impacts in the performance of the team, and therefore of its members. Although there are many research questions for which a personal focus should suffice, if there is no clear justification for using a personal focus on a study of coordination or communication a preference for a team focus should not be controversial.

I claim that for the problem of achieving effective coordination and communication dynamics, an organizational focus is often even better than a team focus, for a variety of reasons. An organizational focus enhances the benefits of a team focus: just as the team informs the individual's habits, culture, and strategies, the organization of which the team is part of informs its own behaviour. Furthermore, my observations suggest that teams are usually a far more porous entity than we make them to be in our literature. People are constantly shuffling around between teams; this leads to a significant cross-pollination and the development of a distinct organizational culture. Events in one team can have significant implications for others in terms of the availability of space, of workers, and of criticality of project success. The products of one team are sometimes used in others, leading to a complex combination of a client-provider and a peer-to-peer relationship. Teams use and are shaped by a web of support roles that the organization provides for several other teams simultaneously. And of course, teams are often not entirely autonomous, but must adhere to decisions, policies, and tools mandated at the organizational level.

An organizational focus allows us to shift the emphasis of our analysis, from the purely technical transformation of requirements to delivered code, to considerations having to do with the survival and viability of the organization. Some software practices and some technical decisions make little sense when analyzed at the team or personal levels, but are comprehensible and beneficial under an organizational perspective \cite{Martin2007}.

Perhaps most importantly, an organizational focus also allows us to set criteria for the success of software projects that are appropriate to the actual perceptions of the members of the organization. Success in a software organization, as in any other kind of natural-open organization, is defined by its members \cite{Scott2007}. They may wish to develop a quality product on time, to expand their market share, to work on an exciting domain, to have a good work-life balance, or to have their firm acquired for a significant profit, among many other possible motivations. The traditional metrics of success in software development (timeliness, low cost, a high rate of function points per effort spent) are appropriate in some cases, but not all. An organizational focus gives the researcher the opportunity to investigate those criteria that are truly considered important by the members of the organization.

It is tempting to continue broadening our scope, to look for the advantages in even larger units of analysis, such as a domain or a culture. There may be good reasons for going down this path, but for studies of coordination and communication, those units seem far too removed from our topic to be of any use.

The studies that support this thesis were executed at the organizational level whenever possible. The main exception is a study for which the unit of analysis was not a social agent, but a unit of work as it moved through the organization \cite{Aranda2009}; even in that study we tried to inform our observations with the organizational context of the units of work.


\section{Empirical Studies}

This section describes in brief the studies that are used in support of the thesis. All of the studies have been published and their details are publicly available, except for the most recent one. Therefore, in the interest of presenting the main arguments and supporting evidence of this thesis in a cohesive manner, I do not elaborate too much on each of these studies here. The descriptions that follow are only meant to provide some context for each of them. They are all extensively discussed in the appendices of this thesis.

This section, then, is only meant to provide some basic information regarding these studies, and to provide a basis for the references in the remainder of the thesis. Whenever a statement is directly supported (or questioned) by a piece of evidence from one of these studies, I describe the details of that piece of evidence in a footnote, such as the one appearing in this page.\footnote{\textbf{Name of Study[, Organization ID]}: Details of the supporting or refuting evidence. The last sentence of such footnotes usually states whether the evidence described is unique, exemplary, or typical in the case studies.}


\subsection{The \emph{Small Firms} Study}
\label{sec:SmallFirmsStudy}

During late 2006 and early 2007, Steve Easterbrook, Greg Wilson, and I conducted a multiple-case exploratory case study of small software firms in the Toronto, Canada area, based on the motivation that these firms form a large part of the software industry but tend to be overlooked by software research. As this was an exploratory case study, our goal was to derive specific hypotheses for further study.

The selection criteria for firms in our study were that the firm was dedicated to software development as a primary activity, that it was small (defined as less than fifty employees), that it had been in operation for at least one year, and (for convenience) that it had offices in the Toronto area.

The study focused on the project management and coordination strategies of those organizations. We collected data from seven of them, on issues such as organizational structure, company size, roles of key staff, line of work, types of customers, financial viability, descriptions of their analysis, sales, negotiation, and development processes, communication of requirements to the rest of the team, documentation, tool support, requirements errors, and misunderstandings between the company and its customers. Table \ref{tab:SmallFirmsDescrip} presents a brief description of these firms, while table \ref{tab:SmallFirmsDetails} summarizes some of their details. Most of our information came from interviews of people ranking high in the hierarchy of the firms and from observations of their work environments. We published an analysis of the diversity of techniques that these firms use to elicit and track requirements \cite{Aranda2007}. Further interviews with people in five additional firms strengthened our belief in the relevance of our findings. Section \ref{sec:SmallFirmsDesign} provides more details on our data collection and analysis techniques.

We identified several major findings in our analysis, having to do with the diversity of work dynamics between our cases, with their apparent cultural cohesion, and with their approaches to requirements analysis and requirements errors. For each of these findings we formulated hypotheses for further investigation. Table \ref{tab:SmallFirmsHyps} summarizes these hypotheses. Some of them were evaluated and refined in the last study, described briefly in section \ref{sec:Contrast}. Appendix \ref{app:SmallFirms} presents more details of this study and each of its cases.


\begin{table}[tbp]
\caption{\label{tab:SmallFirmsDescrip} Description of the cases in the \textbf{Small Firms} study}
\centering
\footnotesize{\begin{tabular}{p{2.4cm}p{11.2cm}}
\hline \hline
\vspace{1pt} \bfseries Pseudonym & \vspace{1pt} \bfseries Description \\
\hline
\vspace{0.5pt} Endosymbiotic & \vspace{0.5pt} Open source firm specializing in applications for the health industry. Its offices are located within the space of their main customer, a general hospital. \\
\hline
\vspace{0.5pt} Agilista & \vspace{0.5pt} Software consulting and development (mostly) for relatively long-term industrial automation projects. \\
\hline
\vspace{0.5pt}Spark & \vspace{0.5pt} Highly specialized, algorithmically complex products and web services for news corporations and publishers. \\
\hline
\vspace{0.5pt} Bespoker & \vspace{0.5pt} Development of enterprise applications for banks, insurance companies, and other large corporations. \\
\hline
\vspace{0.5pt}PhoneOffshore & \vspace{0.5pt} Provider of applications for mobile services with millions of subscribers. Its main clients are large telecommunication corporations. The firm has its headquarters in Toronto and a second development office at an offshore site. \\
\hline
\vspace{0.5pt}Growing Web & \vspace{0.5pt} Web development and consulting start-up focusing on content management applications for a wide variety of customers.\\
\hline
\vspace{0.5pt} Rentcraft & \vspace{0.5pt} Provider of rental management products. \\
\hline
\end{tabular}}
\end{table}

\hyphenpenalty 10000
\begin{landscape}
\begin{table}[tbp]
\caption{\label{tab:SmallFirmsDetails} Some relevant details from the cases in the \textbf{Small Firms} study}
\centering
\footnotesize{\begin{tabular}{p{2.3cm}p{2.3cm}p{2.3cm}p{2.3cm}p{2.3cm}p{2.3cm}p{2.3cm}p{2.3cm}}
\hline \hline
\vspace{0.5pt} \bfseries Attribute & \vspace{0.5pt} \bfseries Endo-symbiotic & \vspace{0.5pt} \bfseries Agilista & \vspace{0.5pt} \bfseries Spark & \vspace{0.5pt} \bfseries Bespoker & \vspace{0.5pt} \bfseries Phone-Offshore & \vspace{0.5pt} \bfseries Growing Web & \vspace{0.5pt} \bfseries Rentcraft \\
\hline
\vspace{0.5pt} Firm Size & \vspace{0.5pt} 7 & \vspace{0.5pt} 4 & \vspace{0.5pt} 19 & \vspace{0.5pt} 40-45 & \vspace{0.5pt} 20-25 & \vspace{0.5pt} 5 & \vspace{0.5pt} 25 \\
\hline
\vspace{0.5pt} Longevity & \vspace{0.5pt} 15 months & \vspace{0.5pt} 13 years & \vspace{0.5pt} 5 years & \vspace{0.5pt} 5 years & \vspace{0.5pt} 7 years & \vspace{0.5pt} 3 years & \vspace{0.5pt} 12 years \\
\hline
\vspace{0.5pt} Customers & \vspace{0.5pt} Hospital & \vspace{0.5pt} Manu-facturing & \vspace{0.5pt} News agencies and publishers & \vspace{0.5pt} Banks and corporations & \vspace{0.5pt} Telecoms & \vspace{0.5pt} Varied (content management) & \vspace{0.5pt} Rental companies \\
\hline
\vspace{0.5pt} Type of offering & \vspace{0.5pt} Product, service & \vspace{0.5pt} Projects & \vspace{0.5pt} Product, service & \vspace{0.5pt} Projects & \vspace{0.5pt} Projects & \vspace{0.5pt} Projects & \vspace{0.5pt} Product \\
\hline
\vspace{0.5pt} Project length / release cycle & \vspace{0.5pt} 1 month & \vspace{0.5pt} 2 weeks & \vspace{0.5pt} 1 year & \vspace{0.5pt} 4 months -- 2 years & \vspace{0.5pt} 6 months & \vspace{0.5pt} 4 hours -- 3 months & \vspace{0.5pt} 9 months -- 1 year \\
\hline
\vspace{0.5pt} Key requirements documents & \vspace{0.5pt} Product backlog & \vspace{0.5pt} Product backlog, user stories & \vspace{0.5pt} \emph{None} & \vspace{0.5pt} Spec, development handbook & \vspace{0.5pt} Statement of work, project plan & \vspace{0.5pt} Cost worksheet, architecture and design document & \vspace{0.5pt} Analysis and estimation, product requirements description \\
\hline
\vspace{0.5pt} Signs of adaptation to niche & \vspace{0.5pt} Co-location with customer & \vspace{0.5pt} \emph{Insufficient data} & \vspace{0.5pt} Year-long negotiation processes & \vspace{0.5pt} \emph{Insufficient data} & \vspace{0.5pt} Homegrown framework & \vspace{0.5pt} Homegrown framework & \vspace{0.5pt} \emph{Insufficient data} \\
\hline
\vspace{0.5pt} Cultural cohesion & \vspace{0.5pt} Previous company & \vspace{0.5pt} Engineering & \vspace{0.5pt} CS PhDs and MScs & \vspace{0.5pt} Previous companies & \vspace{0.5pt} Language and country & \vspace{0.5pt} \emph{None} & \vspace{0.5pt} Previous companies \\
\hline
\vspace{0.5pt} Analyst & \vspace{0.5pt} Founder & \vspace{0.5pt} Founder & \vspace{0.5pt} CEO/CIO & \vspace{0.5pt} Project lead & \vspace{0.5pt} Project lead & \vspace{0.5pt} Founder & \vspace{0.5pt} Product manager \\
\hline
\vspace{0.5pt} Mitigation of requirements errors & \vspace{0.5pt} Monthly demos & \vspace{0.5pt} Iterations & \vspace{0.5pt} Iterations & \vspace{0.5pt} Upfront analysis, iterations & \vspace{0.5pt} Negotiation & \vspace{0.5pt} \emph{None apparent} & \vspace{0.5pt} Upfront analysis, beta testing \\
\hline
\end{tabular}}
\end{table}
\end{landscape}
\hyphenpenalty 1000

\begin{table}[tbp]
\caption{\label{tab:SmallFirmsHyps} Hypotheses derived from the cases in the \textbf{Small Firms} study}
\centering
\footnotesize{\begin{tabular}{p{4.0cm}p{9.6cm}}
\hline \hline
\vspace{1pt} \bfseries Observation & \vspace{1pt} \bfseries Hypothesis \\
\hline
\vspace{0.5pt} Diversity of practices & \vspace{0.5pt} The diversity of practices in small firms can be explained as the result of evolutionary adaptation, as these firms have adapted to a specific niche. \\
\hline
\vspace{0.5pt} Cultural cohesion & \vspace{0.5pt} The choice of practices is irrelevant for small firms with strong cultural cohesion, as the efficiency of team dynamics overrides any benefits based on process. \\
\hline
\vspace{0.5pt} CEO as requirements analyst & \vspace{0.5pt} The skillset needed for successful requirements engineering is a subset of the skillset for successful entrepreneurship. \\
%\hline
\vspace{0.5pt}  & \vspace{0.5pt} Requirements engineering and business strategy are inseparable for small firms. \\
\hline
\vspace{0.5pt} Requirements errors are not catastrophes & \vspace{0.5pt} Small firms that survive their initial phase practice normal design, which greatly decreases the risks associated with requirements engineering. \\
%\hline
\vspace{0.5pt}  & \vspace{0.5pt} Small firms can fix their requirements problems more easily than large firms by virtue of being small. \\
%\hline
\vspace{0.5pt} & \vspace{0.5pt} A single requirements catastrophe will drive a small firm out of business (survivor bias). \\
\hline
\end{tabular}}
\end{table}



\subsection{The \emph{Scientific Groups} Study}
\label{sec:ScientificGroupsStudy}

Next we proceeded by studying small scientific computing groups \cite{Aranda2008b}. We considered scientific computing projects to be those that are software-intensive endeavours integral to the process of discovery and validation of scientific knowledge. Examples include climate models and visualizers of medical phenomena. We excluded trivial programs, proof-of-concepts, and non-trivial software written by scientists for non-scientific purposes, such as derivations of their work for commercialization. Our interest in scientific computing grew out of the realization that its bottleneck is not computing power, as is often assumed, but the skills, coordination, and tool adoption of its developers \cite{Wilson2006}.

We performed a multiple-case exploratory case study of scientific groups located in Toronto that were developing scientific computing software according to our previous definition. A large majority of our cases came from the university environment. All of our data for this study came from interviews of scientists; details of our data collection and analysis can be found in section \ref{sec:ScientificGroupsDesign}. Table \ref{tab:ScientificCases} provides a brief description of our cases.

\begin{table}[tbp]
\caption{\label{tab:ScientificCases} Description of the cases in the \textbf{Scientific Groups} study}
\centering
\footnotesize{\begin{tabular}{p{3.4cm}p{10.2cm}}
\hline \hline
\vspace{1pt} \bfseries Area & \vspace{1pt} \bfseries Description \\
\hline
\vspace{0.5pt} Particle Physics & \vspace{0.5pt} Our interviewees worked with the ATLAS project (the particle accelerator at CERN) to pursue their own research questions and to collaborate with the ATLAS code. \\
\hline
\vspace{0.5pt} Forestry & \vspace{0.5pt} Stochastic-based simulation of forest development under different management strategies. \\
\hline
\vspace{0.5pt} Urban Planning & \vspace{0.5pt} Urban traffic simulator, modelling city-wide effects of public transit policies. \\
\hline
\vspace{0.5pt} Oncology & \vspace{0.5pt} Development of minimally invasive procedures to cure cancer. Software determines the amount of heat to be applied to a tissue. \\
\hline
\vspace{0.5pt} Medical Imaging & \vspace{0.5pt} Processing of MRI images to automatically detect some forms of cancer. \\
\hline
\vspace{0.5pt} Atmospheric Physics & \vspace{0.5pt} Simulator to help explain the formation of atmospheric currents and clouds in Jupiter.\\
\hline
\vspace{0.5pt} Biotechnology & \vspace{0.5pt} Medium-scale geographically distributed project to examine protein interactions visually. \\
\hline
\vspace{0.5pt} Biotechnology & \vspace{0.5pt} Protein interaction database portal. \\
\hline
\vspace{0.5pt} Earth Sciences & \vspace{0.5pt} Graphically-intensive rock and soil engineering tools that analyze the stability of various kinds of structural formations. \\
\hline
\end{tabular}}
\end{table}

Our analysis focused on the relationship between organizational structure and product structure in these groups \cite{Conway1968}; our main discoveries were that both the organizations and their resulting software had very loose boundaries and, broadly, that goal-setting, coordination, and organizational development in this setting proceeded very differently from the for-profit sector. More detailed findings are described in our paper \cite{Aranda2008b} and in Appendix \ref{app:ScientificGroups}.


\subsection{The \emph{IBM} Study}
\label{sec:IBM}

The studies described in the previous two sections correspond to small software organizations. However, we have also explored the question of how to pursue empirical studies of software development in larger organizations. They pose a problem of scale: a single researcher can study groups of a few dozen people and document the major features of their coordination and communication behaviour; the task becomes much more difficult with groups of hundreds or thousands of members.

We discussed this issue twice in position papers. In the earlier \cite{Aranda2006} we explored the feasibility of using Hutchins' Distributed Cognition framework \cite{Hutchins1995} in software development research; we concluded that although it is theoretically promising, its roots in ethnomethodology make it cumbersome and unworkable for non-trivial software development situations, and it requires some level of abstraction to make it work. We proposed the use of Social Network Analysis and Artifact Analysis as possible abstraction strategies that complement well with the framework.

In the latter position paper \cite{Aranda2007b}, which was based on a case study of software development and project management at a release team of IBM, we explored coordination and communication strategies such as the use of management dashboards, status report meetings, and organizational re-structuring. We did this through a combination of interviews, artifact analysis, and observations of release and status meetings. We performed nine interviews to different members of the release team, and we collected detailed observations of seven release and status meetings. Section \ref{sec:IBMDesign} has more details on our data collection and analysis strategies. As a result of the complexity of the interests and data we obtained, we proposed the use of \emph{organizational views}, a concept inspired by Kruchten's \shortcite{Kruchten1995} ``4+1 view model'' of software architecture to make sense of the multiple aspects at play in a complex phenomenon such as large-scale software development. Later work showed us that this proposal was also cumbersome, and some of its aspects unnecessary. My collaboration with Venolia (described in the next section) was in part an attempt to improve over these shortcomings and to be able to address coordination and communication questions of the kind that my thesis asks in large organizations.

Appendix \ref{app:IBM} presents our organizational views model, as well as details of the execution and analysis of the IBM case study that could not be included in our corresponding publication \cite{Aranda2007b} for space reasons.


\subsection{The \emph{Microsoft} Study}
\label{sec:Microsoft}

Gina Venolia and I conducted a study of communication and coordination at Microsoft during the summer of 2008 \cite{Aranda2009}. The study focused on activities related to bug fixing, although we also did a pilot study of coordination for feature development. We executed our field study in two parts. The first was a multiple-case case study of bug histories; the second aimed to validate our case study findings with a survey of software professionals. Table \ref{tab:MicrosoftCases} reports on some of the main characteristics of our ten bug history cases; I describe details of our data collection and analysis techniques in section \ref{sec:MicrosoftDesign}.

We had two main research questions: how is the process of fixing bugs coordinated in large software organizations? And do electronic traces provide a good-enough picture of coordination?

We found that our bug histories were rich, varied, and context dependent. They did not follow a uniform path or lifecycle. Instead of attempting to formulate a process for all bug histories (necessarily incomplete, since we cannot expect to have captured all of the possible details of the process from ten cases in our pool), we chose to describe the menu of coordination patterns that we observed, including some pathological patterns that we had not observed in small organizations. We selected the patterns that seemed to be the most recurrent and those that occurred rarely but had a great impact in the history of a bug. Tables \ref{tab:MicrosoftPatterns1} and \ref{tab:MicrosoftPatterns2} summarize these patterns.

\begin{landscape}
\begin{table}[tbp]
\caption{\label{tab:MicrosoftCases} Summary of the cases in the \textbf{Microsoft} study}
\centering
\footnotesize{\begin{tabular}{p{1.6cm}p{2.6cm}p{2.2cm}p{2.2cm}p{1.6cm}p{1.6cm}p{1.6cm}p{1.6cm}p{1.6cm}p{1.6cm}}
\hline \hline
\vspace{0.5pt} \bfseries Case & \vspace{0.5pt} \bfseries Type of Case & \vspace{0.5pt} \bfseries How was it found & \vspace{0.5pt} \bfseries Resolution & \vspace{0.5pt} \bfseries Direct Agents & \vspace{0.5pt} \bfseries Indirect Agents & \vspace{0.5pt} \bfseries Other Listeners & \vspace{0.5pt} \bfseries Lifespan (days) & \vspace{0.5pt} \bfseries Days with Events & \vspace{0.5pt} \bfseries Events \\
\hline
\vspace{0.5pt} C1 & \vspace{0.5pt} Documentation & \vspace{0.5pt} Ad-hoc test & \vspace{0.5pt} Fixed & \vspace{0.5pt} 6 & \vspace{0.5pt} 4 & \vspace{0.5pt} 179 & \vspace{0.5pt} 320 & \vspace{0.5pt} 12 & \vspace{0.5pt} 19 \\
\hline
\vspace{0.5pt} C2 & \vspace{0.5pt} Code (security) & \vspace{0.5pt} Ad-hoc test & \vspace{0.5pt} Fixed & \vspace{0.5pt} 21 & \vspace{0.5pt} 6 & \vspace{0.5pt} 3 & \vspace{0.5pt} 408 & \vspace{0.5pt} 49 & \vspace{0.5pt} 138 \\
\hline
\vspace{0.5pt} C3 & \vspace{0.5pt} Build test failure & \vspace{0.5pt} Automated & \vspace{0.5pt} Fixed & \vspace{0.5pt} 42 & \vspace{0.5pt} 8 & \vspace{0.5pt} 291 & \vspace{0.5pt} 59 & \vspace{0.5pt} 21 & \vspace{0.5pt} 141 \\
\hline
\vspace{0.5pt} C4 & \vspace{0.5pt} Code (functionality) & \vspace{0.5pt} Ad-hoc test & \vspace{0.5pt} Fixed & \vspace{0.5pt} 6 & \vspace{0.5pt} 1 & \vspace{0.5pt} 0 & \vspace{0.5pt} 7 & \vspace{0.5pt} 5 & \vspace{0.5pt} 16 \\
\hline
\vspace{0.5pt} C5 & \vspace{0.5pt} Code (install) & \vspace{0.5pt} User (beta) & \vspace{0.5pt} By Design & \vspace{0.5pt} 2 & \vspace{0.5pt} 3 & \vspace{0.5pt} 0 & \vspace{0.5pt} 2 & \vspace{0.5pt} 2 & \vspace{0.5pt} 12 \\
\hline
\vspace{0.5pt} C6 & \vspace{0.5pt} Code (functionality) & \vspace{0.5pt} Automated & \vspace{0.5pt} Fixed & \vspace{0.5pt} 2 & \vspace{0.5pt} 4 & \vspace{0.5pt} 11 & \vspace{0.5pt} 29 & \vspace{0.5pt} 6 & \vspace{0.5pt} 20 \\
\hline
\vspace{0.5pt} C7 & \vspace{0.5pt} Build test failure & \vspace{0.5pt} Automated & \vspace{0.5pt} Fixed & \vspace{0.5pt} 6 & \vspace{0.5pt} 7 & \vspace{0.5pt} 197 & \vspace{0.5pt} 14 & \vspace{0.5pt} 6 & \vspace{0.5pt} 34 \\
\hline
\vspace{0.5pt} C8 & \vspace{0.5pt} Code (functionality) & \vspace{0.5pt} Dog food & \vspace{0.5pt} Won't Fix & \vspace{0.5pt} 2 & \vspace{0.5pt} 7 & \vspace{0.5pt} 0 & \vspace{0.5pt} 2 & \vspace{0.5pt} 2 & \vspace{0.5pt} 5 \\
\hline
\vspace{0.5pt} C9 & \vspace{0.5pt} Code (functionality) & \vspace{0.5pt} Automated & \vspace{0.5pt} Not Repro & \vspace{0.5pt} 5 & \vspace{0.5pt} 2 & \vspace{0.5pt} 1 & \vspace{0.5pt} 2 & \vspace{0.5pt} 2 & \vspace{0.5pt} 12 \\
\hline
\vspace{0.5pt} C10 & \vspace{0.5pt} Code (functionality) & \vspace{0.5pt} Escalation & \vspace{0.5pt} Fixed & \vspace{0.5pt} 23 & \vspace{0.5pt} 18 & \vspace{0.5pt} 13 & \vspace{0.5pt} 35 & \vspace{0.5pt} 20 & \vspace{0.5pt} 220 \\
\hline
\end{tabular}}
\end{table}
\end{landscape}

\begin{table}[tbp]
\caption{\label{tab:MicrosoftPatterns1} Communication media and bug database patterns found in the \textbf{Microsoft} study}
\centering
\footnotesize{\begin{tabular}{p{2.4cm}p{12.2cm}}
\hline \hline
\vspace{1pt} \bfseries \emph{Communication media} & \vspace{1pt} \\
\hline
\vspace{0.5pt} Broadcasting emails & \vspace{0.5pt} Sending a manual or automatic notification to a number of mailing lists to inform their members of an event. \\
\hline
\vspace{0.5pt} Shotgun emails & \vspace{0.5pt} Sending an email to a number of mailing lists and individuals in the hope that one of the recipients will have an answer to the current problem. \\
\hline
\vspace{0.5pt} Snowballing threads & \vspace{0.5pt} Adding people to an ever-increasing list of email recipients. \\
\hline
\vspace{0.5pt} Probing for expertise & \vspace{0.5pt} Sending emails to one or few people, not through the ``shotgun'' method, in the hope that they will either have the expertise to assist with a problem or can redirect to somebody that will. \\
\hline
\vspace{0.5pt} Probing for ownership & \vspace{0.5pt} Sending emails to one or few people, not through the ``shotgun'' method, requesting that they accept ownership of the bug or can redirect to somebody that will. \\
\hline
\vspace{0.5pt} Infrequent, direct email & \vspace{0.5pt} Emails sent privately and infrequently among a handful of people. \\
\hline
\vspace{0.5pt} Rapid-fire email & \vspace{0.5pt} Bursts of email activity in private among a few people in the process of troubleshooting the issue. \\
\hline
\vspace{0.5pt} Rapid-fire email in public & \vspace{0.5pt} Like the above, but with tens or hundreds of people copied as recipients of the email thread, most of them unconnected to the issue. \\
\hline
\vspace{0.5pt} IM discussion & \vspace{0.5pt} Using an instant messaging platform to pass along information, troubleshoot, or ping people. \\
\hline
\vspace{0.5pt} Phone & \vspace{0.5pt} Phone conversations used to pass along information, troubleshoot, or ping people. \\
\hline
\hline
\vspace{1pt} \bfseries \emph{Bug database} & \vspace{1pt} \\
\hline
\vspace{0.5pt} Close-reopen & \vspace{0.5pt} A bug that is reopened because it had been incorrectly diagnosed or resolved, or because there is disagreement on its resolution or on the team's ability to postpone addressing it. \\
\hline
\vspace{0.5pt} Follow-up bugs filed & \vspace{0.5pt} Other bugs were found and filed in the process of fixing this one, or a piece of this bug was filed in a different record as follow-up. \\
\hline
\vspace{0.5pt} Forgotten & \vspace{0.5pt} A bug record that goes unnoticed and unattended for long periods. \\
\hline
\end{tabular}}
\end{table}

\begin{table}[tbp]
\caption{\label{tab:MicrosoftPatterns2} Code, meeting, and other patterns found in the \textbf{Microsoft} study}
\centering
\footnotesize{\begin{tabular}{p{2.4cm}p{12.2cm}}
\hline \hline
\vspace{1pt} \bfseries \emph{Working on code} & \vspace{1pt} \\
\hline
\vspace{0.5pt} Code review & \vspace{0.5pt} The fix for this bug was reviewed and approved by at least one peer. \\
\hline
\vspace{0.5pt} Two birds with one stone & \vspace{0.5pt} The fix for this bug also fixed other bugs that had been discovered and filed previously. \\
\hline
\vspace{0.5pt} While we're there & \vspace{0.5pt} The fix for this bug also fixed other bugs that had not been discovered previously. \\
\hline
\hline
\vspace{1pt} \bfseries \emph{Meeting} & \vspace{1pt} \\
\hline
\vspace{0.5pt} Drop by your office & \vspace{0.5pt} Getting a piece of information, or bouncing some ideas regarding the issue, face to face informally with a coworker in a nearby office. \\
\hline
\vspace{0.5pt} Air time in status meeting & \vspace{0.5pt} The issue was discussed in a regular group status meeting. \\
\hline
\vspace{0.5pt} Huddle & \vspace{0.5pt} The issue called for a team meeting exclusively to discuss it. \\
\hline
\vspace{0.5pt} Summit & \vspace{0.5pt} The issue called for a meeting among people from different divisions exclusively to discuss it. \\
\hline
\vspace{0.5pt} Meeting with remote participants & \vspace{0.5pt} Any meeting where at least one member is attending remotely (could be a huddle or summit meeting). \\
\hline
\vspace{0.5pt} Video conferences & \vspace{0.5pt} Any meeting where video was used to communicate with at least one attendee (could be a huddle or summit meeting. \\
\hline
\hline
\vspace{1pt} \bfseries \emph{Other patterns} & \vspace{1pt} \\
\hline
\vspace{0.5pt} Ignored fix/diagnosis & \vspace{0.5pt} A correct diagnosis or fix that was proposed early on and was temporarily ignored by the majority. \\
\hline
\vspace{0.5pt} Ownership avoidance & \vspace{0.5pt} Bouncing ownership of the bug or code. \\
\hline
\vspace{0.5pt} Triaging & \vspace{0.5pt} Discussing and deciding whether this is an issue worth addressing. \\
\hline
\vspace{0.5pt} Referring to the spec & \vspace{0.5pt} At least one concrete and specific reference to a spec, design document, scenario, or vision statement, to provide guidance to solve or settle the issue. \\
\hline
\vspace{0.5pt} Unexpected contribution & \vspace{0.5pt} New information or alternatives that come from people out of the group discussing the issue. \\
\hline
\vspace{0.5pt} Deep collaboration & \vspace{0.5pt} Two or more people working closely (face to face or electronically) and for a sustained period to unravel the issue. \\
\hline
\end{tabular}}
\end{table}


We also found that electronic traces of bugs provide an incomplete and often misleading picture of coordination in software teams, which is an important finding considering the recent exclusive reliance on data-mining research methods in our community. Tables \ref{tab:MicrosoftEvents} and \ref{tab:MicrosoftAgents} give some quantitative indication that the electronic traces of bugs are inadequate summaries of the true efforts in coordination and communication to fix them. Note, however, that these tables only illustrate part of the problem: often, the reasons for the inadequacy of electronic traces were qualitative, more than quantitative. Our paper summarizes these reasons \cite{Aranda2009}. A full discussion of this study, its findings, and implications, can be found in Appendix \ref{app:Microsoft}.

\begin{table}[tbp]
\caption{\label{tab:MicrosoftEvents} Events found at each level of analysis in the \textbf{Microsoft} study}
\centering
\footnotesize{\begin{tabular}{p{1.4cm}p{2.4cm}p{2.4cm}p{2.4cm}p{2.4cm}}
\hline \hline
\vspace{0.5pt} \bfseries Case & \vspace{0.5pt} \bfseries Bug record data (BRD) & \vspace{0.5pt} \bfseries BRD+ Electronic conversations (EC) & \vspace{0.5pt} \bfseries BRD+EC+ Human sense-making (HSM) & \vspace{0.5pt} \bfseries BRD+EC+ HSM+ Direct participant accounts \\
\hline
\vspace{0.5pt} C1 & \vspace{0.5pt} 8 & \vspace{0.5pt} 16 & \vspace{0.5pt} 17 & \vspace{0.5pt} 19 \\
\hline
\vspace{0.5pt} C2 & \vspace{0.5pt} 11 & \vspace{0.5pt} 11 & \vspace{0.5pt} 138 & \vspace{0.5pt} 138 \\
\hline
\vspace{0.5pt} C3 & \vspace{0.5pt} 19 & \vspace{0.5pt} 119 & \vspace{0.5pt} 119 & \vspace{0.5pt} 141 \\
\hline
\vspace{0.5pt} C4 & \vspace{0.5pt} 11 & \vspace{0.5pt} 14 & \vspace{0.5pt} 15 & \vspace{0.5pt} 16 \\
\hline
\vspace{0.5pt} C5 & \vspace{0.5pt} 8 & \vspace{0.5pt} 11 & \vspace{0.5pt} 11 & \vspace{0.5pt} 12 \\
\hline
\vspace{0.5pt} C6 & \vspace{0.5pt} 12 & \vspace{0.5pt} 18 & \vspace{0.5pt} 19 & \vspace{0.5pt} 20 \\
\hline
\vspace{0.5pt} C7 & \vspace{0.5pt} 6 & \vspace{0.5pt} 33 & \vspace{0.5pt} 34 & \vspace{0.5pt} 34 \\
\hline
\vspace{0.5pt} C8 & \vspace{0.5pt} 4 & \vspace{0.5pt} 4 & \vspace{0.5pt} 5 & \vspace{0.5pt} 5 \\
\hline
\vspace{0.5pt} C9 & \vspace{0.5pt} 7 & \vspace{0.5pt} 11 & \vspace{0.5pt} 12 & \vspace{0.5pt} 12 \\
\hline
\vspace{0.5pt} C10 & \vspace{0.5pt} 17 & \vspace{0.5pt} 78 & \vspace{0.5pt} 149 & \vspace{0.5pt} 220 \\
\hline
\end{tabular}}
\end{table}

\begin{table}[tbp]
\caption{\label{tab:MicrosoftAgents} Agents found at each level of analysis in the \textbf{Microsoft} study}
\centering
\footnotesize{\begin{tabular}{p{1.4cm}p{2.4cm}p{2.4cm}p{2.4cm}p{2.4cm}}
\hline \hline
\vspace{0.5pt} \bfseries Case & \vspace{0.5pt} \bfseries Bug record data (BRD) & \vspace{0.5pt} \bfseries BRD+ Electronic conversations (EC) & \vspace{0.5pt} \bfseries BRD+EC+ Human sense-making (HSM) & \vspace{0.5pt} \bfseries BRD+EC+ HSM+ Direct participant accounts \\
\hline
\vspace{0.5pt} C1 & \vspace{0.5pt} 7 & \vspace{0.5pt} 9 & \vspace{0.5pt} 9 & \vspace{0.5pt} 10 \\
\hline
\vspace{0.5pt} C2 & \vspace{0.5pt} 5 & \vspace{0.5pt} 5 & \vspace{0.5pt} 27 & \vspace{0.5pt} 27 \\
\hline
\vspace{0.5pt} C3 & \vspace{0.5pt} 12 & \vspace{0.5pt} 38 & \vspace{0.5pt} 38 & \vspace{0.5pt} 50 \\
\hline
\vspace{0.5pt} C4 & \vspace{0.5pt} 5 & \vspace{0.5pt} 5 & \vspace{0.5pt} 7 & \vspace{0.5pt} 7 \\
\hline
\vspace{0.5pt} C5 & \vspace{0.5pt} 4 & \vspace{0.5pt} 5 & \vspace{0.5pt} 2 & \vspace{0.5pt} 5 \\
\hline
\vspace{0.5pt} C6 & \vspace{0.5pt} 7 & \vspace{0.5pt} 7 & \vspace{0.5pt} 5 & \vspace{0.5pt} 6 \\
\hline
\vspace{0.5pt} C7 & \vspace{0.5pt} 7 & \vspace{0.5pt} 14 & \vspace{0.5pt} 12 & \vspace{0.5pt} 13 \\
\hline
\vspace{0.5pt} C8 & \vspace{0.5pt} 6 & \vspace{0.5pt} 6 & \vspace{0.5pt} 15 & \vspace{0.5pt} 9 \\
\hline
\vspace{0.5pt} C9 & \vspace{0.5pt} 6 & \vspace{0.5pt} 7 & \vspace{0.5pt} 7 & \vspace{0.5pt} 7 \\
\hline
\vspace{0.5pt} C10 & \vspace{0.5pt} 8 & \vspace{0.5pt} 25 & \vspace{0.5pt} 41 & \vspace{0.5pt} 41 \\
\hline
\end{tabular}}
\end{table}



\subsection{The \emph{Contrast} Study}
\label{sec:Contrast}

I executed a final study in response to the previous findings (in particular to those of the Small Firms study). It consisted of a comparison between two similar but contrasting software firms. One of them, Bespoker, had participated in the Small Firms study, and we had reported that it was the one which ``most closely resembled typical views in the requirements literature of how requirements engineering should be done'' \cite{Aranda2007}.

The second firm, which here will be referred to as Saville, was new to our observations. It has a similar size and it services similar kinds of customers as Bespoker, but its software development practices, at least at surface level, are considerably different from those of Bespoker. Saville is a close follower of Extreme Programming \cite{Beck1999} and purportedly tries to do all of its development following XP practices: pair programming, test-driven development, customer on-site, and so forth.

I spent a total of six weeks doing field observations full-time between the two organizations. I interviewed about half of the members of each organization. I also administered a survey to probe the feasibility of developing a group cohesion instrument. Details of my data collection and analysis techniques can be found in section \ref{sec:ContrastDesign}. The results from this study have not yet been published, as I collected my data for this study in the weeks prior to writing this thesis. However, this study has provided some of the richest evidence for the arguments in the following chapter. I include a more extensive discussion of the setting and data from Bespoker and Saville in Appendix \ref{app:Contrast}.
